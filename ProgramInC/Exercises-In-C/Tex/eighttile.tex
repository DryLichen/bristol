\tikzset{word ladder/.style={
  matrix of nodes
  , execute at empty cell={\node[draw=gray,text=gray,fill=white]{0};}
  , nodes in empty cells=false
  , nodes={shape=rectangle, draw=none,fill=none,text=black,minimum width=0.6cm,minimum height=0.35cm,outer sep=0pt,align=center,inner sep=0pt,font=\small}
  , row sep={0.35cm,between origins}
  , column sep={0.6cm,between origins}
},
}

\tikzset{spacehash board/.style={
  matrix of nodes
  , execute at empty cell={\node[draw=gray,text=gray,fill=white]{ };}
  , nodes in empty cells=false
  , nodes={draw=gray,fill=ocre,text=gray,text depth=0.5ex,text height=2ex,text width=1em,outer sep=0pt,align=center,inner sep=0pt}
  , row sep={#1,between origins}
  , column sep={#1,between origins}
},
}

\tikzset{noughtsone board/.style={
  matrix of nodes
  , execute at empty cell={\node[draw=gray,text=gray,fill=white]{0};}
  , nodes in empty cells=false
  , nodes={draw=gray,fill=ocre,text=gray,minimum width=#1,minimum height=#1,outer sep=0pt,align=center,inner sep=0pt}
  , row sep={#1,between origins}
  , column sep={#1,between origins}
},
  noughtsone board/.default=0.5cm
}

\tikzset{bingrid board/.style={
  matrix of nodes
  , execute at empty cell={\node[draw=black,text=black,fill=white]{.};}
  , nodes in empty cells=false
  , nodes={draw=black,fill=white,text=black,minimum width=#1,minimum height=#1,outer sep=0pt,align=center,inner sep=0pt}
  , row sep={#1,between origins}
  , column sep={#1,between origins}
},
  bingrid board/.default=0.75cm
}

\tikzset{twocolour board/.style={
  matrix of nodes
  , execute at empty cell={\node[text=white,fill=white]{+};}
  , nodes in empty cells=false
  , nodes={draw=gray,fill=ocre,minimum width=#1,minimum height=#1,outer sep=0pt,align=center,inner sep=0pt,font=\tiny}
  , text=ocre
  , row sep={#1,between origins}
  , column sep={#1,between origins}
},
  twocolour board/.default=0.5cm
}
\tikzset{twocolour board/.style={
  matrix of nodes
  , execute at empty cell={\node[text=white,fill=white]{+};}
  , nodes in empty cells=false
  , nodes={draw=gray,fill=ocre,minimum width=#1,minimum height=#1,outer sep=0pt,align=center,inner sep=0pt,font=\tiny}
  , text=ocre
  , row sep={#1,between origins}
  , column sep={#1,between origins}
},
  twocolour board/.default=0.5cm
}

%% Candy Crush-style games, colour & label
\tikzset{crushstyle board/.style={
  matrix of nodes
  , nodes={draw=gray,fill=ocre,minimum width=#1,minimum height=#1,outer sep=0pt,align=center,inner sep=0pt,font=\tiny}
  , text=ocre
  , row sep={#1,between origins}
  , column sep={#1,between origins}
},
  crushstyle board/.default=0.5cm
}

%% WaTor-style games, colour & label
\tikzset{watorstyle board/.style={
  matrix of nodes
  , nodes={fill=cyan,minimum width=#1,minimum height=#1,outer sep=0pt,align=center,inner sep=0pt,font=\tiny}
  , text=black
  , row sep={#1,between origins}
  , column sep={#1,between origins}
},
  watorstyle board/.default=0.25cm
}

%% 8-tile style
\tikzset{eighttilestyle board/.style={
  matrix of nodes, ampersand replacement={\&}
  , execute at empty cell={\node[draw=gray,text=gray,fill=gray]{0};}
  , nodes={fill=gray,minimum width=#1,minimum height=#1,outer sep=0pt,align=center,inner sep=0pt,font=\tiny}
  , text=ocre
  , row sep={#1,between origins}
  , column sep={#1,between origins}
},
  eighttilestyle board/.default=0.25cm
}

\tikzset{sixelstyle/.style={
  matrix of nodes, ampersand replacement={\&}
  , nodes={draw=black,fill=gray,text=gray
     %,minimum height=1pt, minimum width=1pt
     %,row sep=1pt, column sep=1pt
}
},
   sixelstyle/.default=0.3em
}

\tikzset{sepsixstyle/.style={
  matrix of nodes, ampersand replacement={\&}
  , nodes={draw=white,fill=gray,text=gray,
     minimum height=1pt, minimum width=1pt,
     row sep=1pt, column sep=1pt}
},
   sixelstyle/.default=0.3em
}

\newcommand{\board}[9]{
\begin{tikzpicture}
\matrix [eighttilestyle board]
{
#1 \& #2 \& #3 \\
#4 \& #5 \& #6 \\
#7 \& #8 \& #9 \\
 };
\end{tikzpicture}
}

\nsection{The 8-Tile Puzzle}

The Chinese 8-Tile Puzzle is a $3 \times 3$ board, with $8$ numbered
tiles in it, and a hole into which neighbouring tiles can move:
\begin{tikzpicture}[every node/.style={anchor=base,text depth=.5ex,text height=2ex,text width=1em,outer sep=0pt,align=center,inner sep=0pt}]
\matrix [matrix of nodes,draw=white,nodes in empty cells]
{
|[fill=ocre,text=black]|1&|[fill=ocre,text=black]|2&|[fill=ocre,text=black]|3 \\
|[fill=ocre,text=black]|4&|[fill=ocre,text=black]|5&|[fill=ocre,text=black]|6 \\
|[fill=ocre,text=black]|7&|[fill=ocre,text=black]|8&|[fill=gray,text=black]|  \\
};
\end{tikzpicture}

\noindent After the next move the board could look like:
\begin{tikzpicture}[every node/.style={anchor=base,text depth=.5ex,text height=2ex,text width=1em,outer sep=0pt,align=center,inner sep=0pt}]
\matrix [matrix of nodes,draw=white,nodes in empty cells]
{
|[fill=ocre,text=black]|1&|[fill=ocre,text=black]|2&|[fill=ocre,text=black]|3 \\
|[fill=ocre,text=black]|4&|[fill=ocre,text=black]|5&|[fill=gray,text=black]|  \\
|[fill=ocre,text=black]|7&|[fill=ocre,text=black]|8&|[fill=ocre,text=black]|6 \\
};
\end{tikzpicture}
or
\begin{tikzpicture}[every node/.style={anchor=base,text depth=.5ex,text height=2ex,text width=1em,outer sep=0pt,align=center,inner sep=0pt}]
\matrix [matrix of nodes,draw=white,nodes in empty cells]
{
|[fill=ocre,text=black]|1&|[fill=ocre,text=black]|2&|[fill=ocre,text=black]|3 \\
|[fill=ocre,text=black]|4&|[fill=ocre,text=black]|5&|[fill=ocre,text=black]|6 \\
|[fill=ocre,text=black]|7&|[fill=gray,text=black]| &|[fill=ocre,text=black]|8  \\
};
\end{tikzpicture}
The problem generally involves the board starting in a random state, and the user
returning the board to the `ordered' $"12345678"$ state.

In this problem, a solution could be found in many different ways; the solution could be recursive,
or you could implement a queue to perform a breadth-first seach, or something more complex allowing
a depth-first search to measure `how close' (in some sense) it is to the correct solution. 

\begin{exercise}
\label{ex:basic8tile}
Read in a board using \verb^argv[1]^, e.g.:
\begin{codesnippet}
$ 8tile "513276 48"
\end{codesnippet}

To do this you will use a list of boards. The initial board is
put into this list. Each board in the list is, in turn, read from the
list and all possible moves from that board added into the list. The
next board is taken, and all its resulting boards are added, and so
on.  This is, essentially, a queue.

However, one problem with is that repeated boards may be put into the queue and
`cycles' occur.  This soon creates an explosively large number of
boards (several million).  You can solve this by only adding a board
into the queue if an identical one does not already exisit in the queue.
A linear search is acceptable for this task of identifying duplicates.
Each structure in the queue will contain (amongst other things)
a board and a record of its parent board, i.e. the board that it was
created from.

Be advised that a solution requiring as `few' as $20$ moves may take
$10$'s of minutes to compute. If the search is successful, display the solution
to the screen using plain-text.

Use the method described above and only this one. Use a static data structure to acheive
this (arrays) and {\bf not} a dynamic method such as linked-lists; a (large) $1D$ array
of structures is acceptable. Because this array needs to be
so large, it's best to declare it in \verb^main()^ using something like:
\begin{codesnippet}
static boards[NUMBOARDS];
\end{codesnippet}

\end{exercise}

\begin{exercise}
Repeat Exercise~\ref{ex:basic8tile}, but use SDL for the ouput rather than plain-text.
\end{exercise}

\begin{exercise}
\label{ex:ll8tile}
Repeat Exercise~\ref{ex:basic8tile}, but using a dynamic (linked-list), so that you never have to make any assumptions about the maximum numbers of boards stored.
\end{exercise}

\begin{exercise}
Repeat Exercise~\ref{ex:ll8tile}, but using a $5 \times 5$ board instead.
\end{exercise}
