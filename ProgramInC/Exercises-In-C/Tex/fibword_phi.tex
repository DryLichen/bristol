\nsection{Fibonacci Words Using $\phi$}
\label{sec:fibword_phi}

\newexercise{2022}

\wwwurl{https://en.wikipedia.org/wiki/Fibonacci\_word}

A {\em Fibonacci Word} is simply a particular (infinite) sequence of digits:
\begin{verbatim}
010010100100101001010010010100100101001010010010100101001001010010010...
\end{verbatim}
and shouldn't be confused with the related, but totally different {\em Fibonacci Sequence} (which we'll see later in the lecture series).

You start by defining $S_0$ to be the sequence "0", and $S_1$ to be 
"01". Each subsequent part of the sequence is made by simply concatenating
the previous two, so $S_2 = 010$ and $S_3 = 01001$ and so on.

Perhaps somewhat surprisingly, the $n^{th}$-digit can be
predicted without computing the rest of the sequence, using the equation~:
\begin{equation}
\label{eq:fibphi_math}
2 + \lfloor n \phi \rfloor - \lfloor (n+1) \phi \rfloor
\end{equation}
or, maybe more understandably~:
\begin{equation}
\label{eq:fibphi}
2 + \operatorname{floor}(n \phi) - \operatorname{floor}((n+1) \phi)
\end{equation}
Here \verb^floor()^ is the function that
rounds a number down to its lowest integer part (e.g. $1.815$ becomes
$1.0$), and where $\phi$ is the golden ratio $=1.61803398875...$.
Note also that the sequence is assumed to begin at the $n=1$, and not 
at zero which a programmer might assume!

\begin{exercise}
\label{ex:fibword_phi}
Write the function:
\begin{codesnippet}
bool fibword_phi(int n)
\end{codesnippet}
which returns to $n^{th}$-digit of the sequence (which begins at $n=1$)
using: an approximation of $\phi$, the function \verb^floor()^ found in 
the mathematics library, and Equation~\ref{eq:fibphi}.
\end{exercise}

[Since $\phi$ can only be stored approximately on a computer,
for larger numbers of $n$, this equation doesn't hold. Exactly
where this happens depends upon how closely you have approximated~$\phi$.
We will eplore this later in Exercise~\ref{ex_fibword_s_vs_p}.]
