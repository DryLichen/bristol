\nsection{SDL - Intro}

Many programming languages have no inherent graphics capabilities.
To get windows to appear on the screen, or to draw lines and shapes,
you need to make use of an external library. Here we use SDL\footnote{
actually, we are using the most recent version SDL2, which is installed
on all the lab machines}, a cross-platform library providing the user with
(amongst other things) such graphical capabilities.

\wwwurl{https://www.libsdl.org/}

The use of SDL is, unsurprisingly, non-trival, so some simple wrapper
files have been created (\verb^neillsdl2.c^ and \verb^neillsdl2.h^).
These give you some simple functions to initialise a window, draw
rectangles, wait for the user to press a key etc.

An example program using this functionality is
provided in a file \verb^demo_neillsdl2.c^.

This program initialises a window, then sits in a loop, drawing
randomly positioned and coloured squares, until the
user presses the mouse or a key. 

\begin{exercise}
Using the \verb^Makefile^ provided, compile and run this program.
Now adapt it, so that the colour of the boxes displayed are all
(random) shades of red.

SDL is already installed on lab machines. At home, if you're using a
ubuntu-style linux machine, use: \verb^sudo apt install libsdl2-dev^
to install it.
\end{exercise}
