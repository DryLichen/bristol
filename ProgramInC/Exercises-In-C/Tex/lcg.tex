\nsection{Linear Congruent Generator}

One simple way to generate `random' numbers is via a Linear Congruent Generator which
might look something like this:
\begin{codesnippet}
   int seed = 0;
   /* Linear Congruential Generator */
   for(i=0; i<LOOPS; i++){
      seed = (A*seed + C) % M;
      /* Seed now contains your new random number */
   }
\end{codesnippet}
here, $A$, $C$ and $M$ are constants defined in the code. All of these pseudo-generators have a period of repetition
that is shorter than $M$. For instance, with $A$ set to $9$, $C$ set to $5$ and $M$ set to $11$ the sequence of numbers is~:
\begin{terminaloutput}
5
6
4
8
0
5
\end{terminaloutput}
and so repeats after $5$ numbers (period equals $5$).

\begin{exercise}
Adapt the above program so that it prints the period of the LCG, where you've \verb^#defined^ the constants $A, C$ and $M$ and \verb^seed^ always begins at zero. For the constants described above, the program would output $5$. For $A=7$, $C=5$ and $M=11$ it will output $10$.
\end{exercise}
