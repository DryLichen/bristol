\nsection{Sierpinski Carpet}

\wwwurl{en.wikipedia.org/wiki/Sierpinski_carpet}

The square is cut into 9 congruent subsquares in a 3-by-3 grid, and
the central subsquare is removed. The same procedure is then applied
recursively to the remaining 8 subsquares, ad infinitum.


\wwwurl{http://www.evilmadscientist.com/2008/sierpinski-cookies/}

\begin{exercise}
Write a program that, in plain text, produces a Sierpinski Carpet.
\end{exercise}

\begin{exercise}
Write a program that, using \verb^neillsimplescreen^,
produces a Sierpinski Carpet.
\end{exercise}

\begin{exercise}
Write a program that, using SDL, produces a Sierpinski Carpet.
\end{exercise}
