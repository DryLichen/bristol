\nsection{Wireworld}

Wireworld is a cellular automaton due to Brian Silverman, formed from a 2D grid
of cells, designed to simulate digital electronics. Each cell can be in one
of four states, either `empty', `electron head', `electron tail' or `copper' (or `conductor').

The next generation of the cells follows the rules, where $n$ is the number of electron heads found in the 8-surrounding cells:
\begin{itemize}
\item empty $->$ empty
\item electron head $->$ electron tail
\item electron tail $->$ copper
\item copper $->$ electron head if $n==1$ or $n==2$
\item copper $->$ copper otherwise
\end{itemize}
\noindent See also:\\
\wwwurl{https://en.wikipedia.org/wiki/Wireworld}
\wwwurl{http://www.heise.ws/fourticklogic.html}

{\samepage
\begin{exercise}
\label{ex:wirew}
Write a program which is run using the \verb^argc^ and
\verb^argv^ parameters to \verb^main^. The usage is
as follows~:
\begin{terminaloutput}
$ wireworld wirefile.txt
\end{terminaloutput}
where \verb^wirefile.txt^ is a file specifying the initial
state of the board. This file codes empty cells as ` ', 
heads as `H', tails as `t' and copper as `c'.
Display the board for $1000$ generations using plain text.
You may assume that the grid is always $40$ cells by $40$

\end{exercise}
}
