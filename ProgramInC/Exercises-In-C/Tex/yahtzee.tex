\newcommand{\fivedice}[5]{\epsdice[white]{#1}\epsdice[white]{#2}\epsdice[white]{#3}\epsdice[white]{#4}\epsdice[white]{#5}}
\nsection{Yahtzee}


The game of Yahtzee is a game played with five dice, and you
try to obtain certain `hands'.
In a similar way to poker, these hands could include a {\it Full House} (two dice are the same, and another three are the same), e.g.:

\fivedice{6}{6}{1}{6}{1}
or  
\fivedice{4}{1}{1}{1}{4}

\noindent or another possible hand is {\it Four-of-a-Kind}, e.g.:
\fivedice{3}{3}{1}{3}{3}
or 
\fivedice{5}{5}{5}{2}{5}
(but not
\fivedice{3}{3}{3}{3}{3}
which is {\it Five-of-a-Kind})

A little mathematics tells use that the probability of these two hands should be $3.85\%$ and $1.93\%$ respectively.
\begin{exercise}
Complete the following program which simulates, by analysing a large number of random dice rolls, the probabilty of each of these two hands.
The five dice of the hand are stored in an array, and to facilitate deciding which hand you've got, a histogram is computed to say how often a \epsdice[white]{1} occurs in the hand, how often a \epsdice[white]{2} occurs and so one. A {\it Full-House} occurs when both a $2$ and a $3$ occurs in the histogram; a {\it Four-of-a-Kind} occurs when there is a $4$ somewhere in the histogram.
{\small
\begin{codesnippet}
#include <stdio.h>
#include <stdbool.h>
#include <stdlib.h>
#include <assert.h>

#define MAXTHROW 6
#define NUMDICE 5
#define TESTS 10000000

/* Fill the array d with random numbers 1..6 */
void randomthrow(int d[NUMDICE]);
/* Decide if the number n occurs anywhere in the histogram h */
bool histo_has(const int h[MAXTHROW], const int n);
/* Compute a histogram, given a dice hand */
void makehist(const int d[NUMDICE], int h[MAXTHROW]);
/* Check that the histograms h1 & h2 are the same */
bool hists_same(const int h1[MAXTHROW], const int h2[MAXTHROW]);
/* Does this hand have 2 lots of one number and 3 lots of another */
bool isfullhouse(const int d[NUMDICE]);
/* Does this hand have 4 lots of one number and 1 of another ? */
bool is4ofakind(const int d[NUMDICE]);
/* Do some testing of the functions required */
void test(void);

int main(void)
{

   int dice[NUMDICE];
   int k4 = 0;
   int fh = 0;
   int i;

   test();
   for(i=0; i<TESTS; i++){
      randomthrow(dice);
      if(isfullhouse(dice)){
        fh++;
      }
      if(is4ofakind(dice)){
        k4++;
      }
   }
   printf("FH=%.2f%% 4oK=%.2f%%\n", (double)fh*100.0/(double)TESTS,
                                       (double)k4*100.0/(double)TESTS);

   return 0;
}
\end{codesnippet}
}

the test function should look like:
{\small
\begin{codesnippet}
void test(void)
{
   int i, j;
   int h1[MAXTHROW] = {0,0,0,0,0,5}; /* 5-of-a-kind */
   int h2[MAXTHROW] = {0,0,5,0,0,0}; /* 5-of-a-kind */
   int h3[MAXTHROW] = {0,2,0,0,3,0}; /* Full-House  */
   int h4[MAXTHROW] = {2,0,3,0,0,0}; /* Full-House  */
   int h5[MAXTHROW] = {3,2,0,0,0,0}; /* Histo of d1 */
   int h6[MAXTHROW] = {0,0,0,2,3,0}; /* Histo of d2 */
   int h7[MAXTHROW] = {4,1,0,0,0,0}; /* Histo of d3 */
   int h8[MAXTHROW] = {0,0,0,0,1,4}; /* Histo of d4 */
   int h9[MAXTHROW];                 /* Temp        */
   int d1[NUMDICE]  = {1,1,1,2,2};   /* Full House  */
   int d2[NUMDICE]  = {5,4,5,4,5};   /* Full House  */
   int d3[NUMDICE]  = {2,1,1,1,1};   /* 4-of-a-kind */
   int d4[NUMDICE]  = {6,6,6,6,5};   /* 4-of-a-kind */
   int d5[NUMDICE]  = {6,6,6,6,5};   /* Temp        */

   /* Tricky to test a random throw ...  */
   for(i=0; i<100; i++){
      randomthrow(d5);
      for(j=0; j<NUMDICE; j++){
         assert((d5[j] >= 1) && (d5[j] <= MAXTHROW));
      }
      makehist(d5,h9);
      for(j=0; j<MAXTHROW; j++){
         assert((h9[j] >= 0) && (h9[j] <= NUMDICE));
      }

   }
   assert(hists_same(h1,h1)==true);
   assert(hists_same(h5,h5)==true);
   assert(hists_same(h1,h2)==false);
   assert(hists_same(h4,h5)==false);
   makehist(d1,h9); assert(hists_same(h9,h5));
   makehist(d2,h9); assert(hists_same(h9,h6));
   makehist(d3,h9); assert(hists_same(h9,h7));
   makehist(d4,h9); assert(hists_same(h9,h8));
   assert(histo_has(h1,5)==true);
   assert(histo_has(h1,0)==true);
   assert(histo_has(h1,2)==false);
   assert(histo_has(h2,5)==true);
   assert(histo_has(h2,0)==true);
   assert(histo_has(h2,2)==false);
   assert(histo_has(h3,3)==true);
   assert(histo_has(h3,2)==true);
   assert(histo_has(h3,1)==false);
   assert(histo_has(h4,3)==true);
   assert(histo_has(h4,2)==true);
   assert(histo_has(h4,1)==false);
   assert(isfullhouse(d1)==true);
   assert(isfullhouse(d2)==true);
   assert(is4ofakind(d1)==false);
   assert(is4ofakind(d2)==false);
   assert(isfullhouse(d3)==false);
   assert(isfullhouse(d4)==false);
   assert(is4ofakind(d3)==true);
   assert(is4ofakind(d4)==true);
}
\end{codesnippet}
}
\end{exercise}
