\section{G: Prettifying (New Types and Aliasing)}

\begin{frame}[fragile]
\frametitle{Enumerated Types}
\begin{columns}
\begin{column}{0.45\textwidth}
{\small
\begin{verbatim}
enum day { sun, mon, tue, wed, thu, fri, sat};
\end{verbatim}
}

\begin{itemize}[<+->]
\item This creates a user-defined {\bf type} \verb^enum day^.
\item The enumerators are constants of type \verb^int^.
\item By default the first (\verb^sun^) has the value \verb^0^,
the second has the value \verb^1^ and so on.
\end{itemize}
\end{column}

\pause
\begin{column}{0.45\textwidth}
\begin{itemize}[<+->]
\item An example of their use:
\begin{verbatim}
enum day d1;
 . . .
d1 = fri;
\end{verbatim}
\item The default numbering may be changed as well:
{\small
\begin{verbatim}
enum fruit{apple=7, pear, orange=3, lemon};
\end{verbatim}
}
\item Use enumerated types as constants to aid readability -
they are self-documenting.
\item Declare them in a header (\verb^.h^) file.
\item Note that the type is \verb^enum day^; the
keyword \verb^enum^ is not enough.
\end{itemize}
\end{column}


\end{columns}
\end{frame}

\begin{frame}[fragile]
\frametitle{Typedefs}

\begin{itemize}[<+->]
\item Sometimes it is useful to associate a particular name with
a certain type, e.g.:
\begin{verbatim}
typedef int colour;
\end{verbatim}
\item Now the type \verb^colour^ is synonymous with the type \verb^int^.
\item Makes code self-documenting.
\item Helps to control complexity when programmers are building
complicated or lengthy user-defined types (See Structures later).
\end{itemize}
\end{frame}

\begin{frame}[fragile]
\frametitle{Combining {\tt typedef}s and {\tt enum}s}
\begin{columns}

\begin{column}{0.45\textwidth}
\begin{itemize}[<+->]
\item Often \verb^typedef^'s are used in conjunction with enumerated types:
\end{itemize}

\lstinputlisting[style=basicc,linerange={1-16},numbers=none]{../Code/ChapG/days.c}
\end{column}

\begin{column}{0.45\textwidth}
\lstinputlisting[style=basicc,linerange={17-100},numbers=none]{../Code/ChapG/days.c}
\end{column}

\end{columns}
\end{frame}

\begin{frame}[fragile]
\frametitle{Style}

\begin{verbatim}
enum veg {beet, carrot, pea};
typedef enum veg veg;
veg v1, v2;
v1 = carrot;
\end{verbatim}
\begin{itemize}[<+->]
\item We can combine the two operations into one:
{\small
\begin{verbatim}
typedef enum veg {beet,carrot,pea} veg;
veg v1, v2;
v1 = carrot;
\end{verbatim}
}
\item Assigning:
\begin{verbatim}
v1 = 10;
\end{verbatim}
is very poor programming style !
\end{itemize}
\end{frame}


\begin{frame}[fragile]
\frametitle{Booleans}
\begin{columns}

\begin{column}{0.45\textwidth}
\begin{itemize}[<+->]
\item Before C99 you might have been tempted to define your own Boolean type:
\end{itemize}
\lstinputlisting[style=basicc]{../Code/ChapG/bool_old.c}
\outputlisting{../Code/ChapG/bool_old.autoout}
\end{column}

\pause
\begin{column}{0.45\textwidth}
\begin{itemize}[<+->]
\item However, we can just use \verb^#include <stdbool.h>^
\end{itemize}
\lstinputlisting[style=basicc]{../Code/ChapG/bool.c}
\outputlisting{../Code/ChapG/bool.autoout}
\end{column}

\end{columns}
\end{frame}


\begin{frame}[fragile]
\frametitle{Fever}
Rewrite/complete this code using \verb^typedefs^ and \verb^enum^s to
create self-documenting code in any manner you wish.
\lstinputlisting[style=basicc]{../Code/ChapG/fvr.c}
\end{frame}

